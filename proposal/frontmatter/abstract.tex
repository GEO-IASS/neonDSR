% Write in only the text of your abstract, all the extra heading jargon is automatically taken care of
\begin{abstract}
Ecological sciences benefit from the huge diversity of plant species which play an important role in large scale ecological aspects such as global warming, land cover change, CO\textsuperscript{2} emission, invasive species, fire hazard, and etc. State-of-the-art species classification techniques utilize remote sensing data such as hyperspectral and LiDAR, however this task involves plenty of field data collection which is both highly time consuming, costly and can only be accomplished by ecological experts. Among thousands of the most commomly found plant species there is huge similarities between them from a remote sensing point of view which makes the task of species classification very daunting; therefore we see a whole body of literature specifically dedicated to this issue which is yet far from real world scenarios with thousands of possible species. While this is an indicator of the importance and complexity of the issue, little has been done to tackle the problem from a computational point of view harnessing the power of "big data". Periodic airborne campaigns can generate terrabytes of data on vast swaths of land. To tackle these problems we propose to use probabilistic knowledge bases and deep learning both of which work best when there is lots and lots of data. Probabilistic knowledge base captures ecological expert knowledge in terms of probabilistic rules, which will be maped to remote sensing data and used to infer new facts and therefore enhance species classification accuracy. Deep learning on the other hand as a semi-supervised algorithm will benefit from the vast amounts of data available and capture intrinsic features of data through its layered architecture and thus help in reducing the amount of labeled data required. 
\end{abstract}
