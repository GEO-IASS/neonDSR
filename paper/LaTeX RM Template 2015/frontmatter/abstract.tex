% Write in only the text of your abstract, all the extra heading jargon is automatically taken care of
\begin{abstract}
Ecological sciences benefit from the huge diversity of plant species which play an important role in large scale ecological aspects such as global warming, land cover change, CO\textsuperscript{2} emission, invasive species, fire hazard, and etc. While this is an indicator of the complexity of the issue, little has been done to tackle the problem from a computational point of view. Current techniques involve manual measurements in local observatory sites where filed data collection is highly time consuming and costly. Remote sensing technologies such as hyperspectral and LiDAR expand the areas under coverage but measurements remain local and labor-intensive. In this proposal we focus on species classification from remote sensing data. Species classification is specifically hard due to the large number of species and their similarities in terms of remote sensing features. To tackle these problems we use probabilistic knowledge bases and deep learning. Probabilistic knowledge base captures ecological expert knowledge in terms of probabilistic rules, which will be further used to infer new facts based on remote sensing data and enhance  classification accuracy. Deep learning, as a semi-supervised algorithm will help in reducing the amount of field data required. Layers of the network capture intrinsic features of data through pre-training and avoid the main problem of other machine learning techniques which is extracting good features. Fine-tuning network parameters needs much less labeled data to perform classification thus avoiding costs. The final issue that we address is that periodic airborne campaigns produces terabytes of data and there is no cohesive system addressing this scalabily. We propose to use custom distributed array databases for this purpose. 
\end{abstract}
