\chapter{CONCLUSION}

Acquiring information for a risk averse manager is never a trivial
task. One might think the imposed risk from motivating another task
would naturally force the manager to collect more information. But
the contract design is subtle. If the imposed risk on the manager is
too small, he will not acquire information, but just invest in the
project; if the imposed risk is too high, the manager will not do
the work either, but just
forgo the project.\footnote{%
\hangpara{.31in}{1}\footnotesize{Laux [2004] pointed out that
motivating the manager to implement another task cannot
automatically provide the manager the right incentive to collect
information. It is always necessary to motivate information
acquisition explicitly.}} This paper shows that auditing, when
conducted properly, can help create efficient incentives for the
manager to acquire information.

An audit of the manager's acquired information is beneficial because
it aligns the manager's incentives in acquiring useful information
and in making a proper investment decision. But if the board of
directors cannot commit not to use the disclosed information to
renegotiate the initial contract, an extensive audit may exacerbate
the control problem. Therefore, if the audit technology is not
highly effective in identifying misstatements, the auditor may only
want to verify whether the manager's report is consistent with his
investment decision, but allow the manager to keep the finer details
private. This arrangement is beneficial because it also reins in the
self-interested behavior coming from the owner's side. Together with
the manager's behavior, it depicts an interesting balance in
equilibrium.

Financial reports contain information that is useful for future
decision-making. The consequences of past decisions are also
recorded in financial reports. The auditor verifies the reported
information thus serves as a monitoring device of managers'
decisions. It is important for the auditor to understand the
manager's dynamic decisions making. The manager's decision in
financial reporting is correlated with his decision in productive
actions, which in turn affects the underlying resource allocation.
Ex post uses of reported information influence the manager's ex ante
incentives to acquire information and make a proper investment
decision. It is also important to recognize the economic
consequences of standard setting. The rules on financial reporting
and auditing change the preparers' behavior. Standard-setters are
not regulating nature but rational economic agents.

The results echo the line of literature that provides explanations
for earnings management based on the effect of renegotiation (a
violation of the
Revelation Principle's assumptions).\footnote{%
See Demski and Frimor [1999], Christensen, Demski and Frimor [2002],
Christensen, Feltham and Sabac [2004] and Gigler and Hemmer [2004].}
My analysis focuses on how renegotiation affects the manager's
investment behavior when the audited information is endogenously
acquired. Audit technology determines how much information is to be
disclosed. In contrast, a perfect audit in my model results in a
first-best scenario and the manager's information should be
disclosed. The intuition is that the manager's shirking means no
useful information is produced and thus will be detected in the
audit. The adverse effect of renegotiation is amplified only when
the audit technology is relatively ineffective so using the
auditor's report alone is not informative enough to provide
incentives for the agent to work. My analysis emphasizes the
subtlety of the audit function and therefore the solution is
\textquotedblleft interior\textquotedblright
---motivating accurate financial reports may or may not be efficient.
Auditors' judgments are the centerpiece.

Some argue that we impose too much responsibility on the auditor.
The auditor's job is to \textquotedblleft check whether a reported
number is correct\textquotedblright . The auditor does not ask why
and how the number is generated. As the auditor assesses audit risk
and materiality before performing substantive tests of
transactions,\ however, she is concerned about management. SAB 99
advises the auditor to investigate the manager's incentives
carefully as opposed to setting some mechanical materiality
threshold. Accounting firms hire experts to audit R\&D contracts
because they have superior knowledge to evaluate the manager's
performance. Auditing is not a simple task in that it requires the
auditor formulate judgments. The deeper the auditor understands
managerial decisions, the easier the auditor reaches a correct
conclusion.

Some worry about the auditor's incentives if she is provided more
discretion. That is a valid concern. There is another round of
incentive problems. One problem is how the PCAOB evaluates the
auditor's work when their opinion is at odds with the auditor's
judgments. Auditor's exposure to legal liability forces standard
setters to consider simple, rules-based standards that permits less
discretion. The audit fees, market competition, etc., all influence
the auditor's behavior. A model with multiple players would be more
appropriate to address these questions. But our model provides a
salient structure of the audit function. More importantly, we point
out the gap in the understanding of financial reporting and
auditing. If financial reporting is a sophisticated communication
process, auditing should help to serve this goal. We first provide a
benchmark where an ideal auditor should perform, then we search for
feasible mechanisms to induce the auditor to perform as we hope.
After all, the questions boil down to the trade off between
revealing information enforced by an auditor and the resulting
concerns for efficiency. The main message from my study is well
reflected here: the optimal auditor's choice depends on the context
that creates the incentive nexus and there is no panacea for all the
reporting issues. Auditors are expected to rely on judgments to
deliver high-quality work.

Besides revelation mechanisms, there are other mechanisms that may
make the discovered information useful. Contract renegotiation takes
into consideration newly discovered information. Thus the manager's
investment incentives are better aligned with the current situation.
In this way, the spillover effect between information acquisition
and investment decisions can be isolated. Efficiency is strictly
improved. However, the result hinges on the assumption that there is
no additional cost to invest in the project. An extension would drop
this assumption and examine the spillover effect between the two
tasks. We might find that interim contract renegotiation can disrupt
the synergy between the two tasks.
