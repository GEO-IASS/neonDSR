\chapter{QUANTUM CHEMISTRY}%
\label{ch_QC}

I've borrowed a few examples of equations I've run across to illustrate how the equations should look in the thesis or dissertation. \cite{Agarwal06}

\section{The Electronic Problem}

The starting point of any discussion of quantum mechanics is the non-relativistic, time-dependent Shr\"{o}dinger equation~\cite{Sza89},

\begin{gather}
{\it i}\hbar\frac{\partial}{\partial t}\Psi(r,t)=\hat{H}\Psi(r,t)\hspace{0.2cm},
\end{gather}

\noindent where $\hbar$ is Planck's constant, $\Psi(r,t)$ is the wave function of the quantum system and $\hat{H}$ is the Hamiltonian, which is an operator that contains a kinetic energy term and a potential energy term.

When we restrict the wave function to be a product of a function of time and a function of space, as, for example, is the case when the potential energy term of the Hamiltonian is independent of time, the time-independent Scrh\"{o}dinger equation can be expressed as:

\begin{gather}
\hat{H}\Psi=E\Psi\hspace{0.2cm}.
\end{gather}

The specific form of the Hamiltonian for molecules is:

\begin{gather}
\label{molH}
{\begin{split}
\hat{H} =-\sum_{A=1}^{M} \frac{1}{2M_{A}} \nabla^{2}_{A} - \sum_{i=1}^{N} \frac{1}{2} \nabla^{2}_{i} - \sum_{i=1}^{N}\sum_{A=1}^{M}\frac{Z_{A}}{r_{iA}} \\
 + \sum_{i=1}^{N}\sum_{j>i}^{N} \frac{1}{r_{ij}} + \sum_{A=1}^{M}\sum_{B>A}^{M} \frac{Z_{A} Z_{B}}{R_{A B}}\hspace{0.2cm},
\end{split}}
\end{gather}

\noindent where $A$ and $B$, etc., label nuclei, and {\it i}, {\it j}, etc., label electrons, $Z$ is the atomic number, and Hartree atomic units ($\hbar = e= m_e = 1$) have been used.


The Born-Oppenheimer approximation, which is a useful and central approximation in quantum chemistry, separates electronic and nuclear motions. Assuming that the nuclei are fixed (since nuclei are much heavier than the electrons), the nuclear kinetic energy term, which is the first term in equation \eqref{molH}, can be neglected, and the repulsion between nuclei, the third term of equation \eqref{molH}, is a constant. This approximation leads to an electronic Hamiltonian,

\begin{gather}
\label{elecH}
\hat{H}_{elec}=- \sum_{i=1}^{N} \frac{1}{2} \nabla^{2}_{i} - \sum_{i=1}^{N}\sum_{A=1}^{M}\frac{Z_{A}}{r_{iA}} + \sum_{i=1}^{N}\sum_{j>i}^{N} \frac{1}{r_{ij}}\hspace{0.2cm},
\end{gather}

\noindent and the Schr\"{o}dinger equation becomes:

\begin{gather}
\hat{H}_{elec}\Psi_{elec}=E_{elec}\Psi_{elec}\hspace{0.2cm}.
\end{gather}

\section{Hartree-Fock Approximation}

Except for the simple case of $H_2^+$, molecules are many-electron problems and determining accurate molecular orbitals, which are the eigenfunctions of the Schr\"{o}dinger equation for a molecule, has been the main task of quantum chemists for many years. Approximate methods have been developed to solve the Schr\"{o}dinger equation, since it is intractable computationally to find the exact solution for a many-electron system. One approximate method that is used frequently to solve the Schr\"{o}dinger equation is based on Hartree-Fock theory. In the present work, Hartree-Fock is not the main method used, but it is crucial to introduce it to explain the methods on which this work is based.

Consider a trial function in the form of a single N-electron Slater-determinant, which obeys the Pauli exclusion principle,

\begin{gather}
\mid \Psi \rangle = \hat{O}\hat{A}\mid \phi_1\phi_2 ...\phi_{\alpha}... \phi_N\rangle\hspace{0.2cm}.
\end{gather}

Here $\hat{O}$ is the spin projector operator that ensures that the wave function remains an eigenfunction of the spin-squared operator ($\hat{S}^2$), $\hat{A}$ is the antisymmetrizer, $\phi_{\alpha}$ is a one-electron wave function that represents the molecular orbital, and Dirac notation has been adopted.

The molecular orbitals can be expanded as a linear combination of atomic orbitals $\psi_{\alpha}$ ,

\begin{gather}
\phi_i=\sum_u\chi_uC_{ui}=XC_i\hspace{0.2cm},
\end{gather}

\noindent which constitute the basis set for the calculation.

$\Psi$ is varied with respect to $C$ following the variational principle to minimize the expectation value of the electronic Hamiltonian, $\hat{H}$ (where the electronic subscript has been dropped for simplicity) to give the following expression for the effective one-particle Fock operator, $f$,

\begin{gather}
f\mid\phi_i\rangle=[h+\sum_{j=1}^{N}J_{j}-K_{j}]\mid\phi_i\rangle=\sum_{j=1}^{N}\epsilon_{ji}\mid\phi_j\rangle\hspace{0.2cm},
\end{gather}

\noindent where $h$ represents the first two terms of equation \eqref{elecH} and $J$ and $K$ are the coulomb and exchange operators respectively. Using a unitary basis to diagonalize the Hermitian matrix, $\epsilon$, with matrix elements $\epsilon_{ji}$ yields the canonical Hartree-Fock equation,

\begin{gather}
f\phi_i=\epsilon_i\phi_i\hspace{0.2cm}.
\end{gather}

From this equation the following generalized-eigenvalue expression can be obtained,

\begin{gather}
FC=SCE\hspace{0.2cm},
\end{gather}

\noindent where $F$ is the Fock matrix, $C$ is a square matrix containing the molecular orbital coefficients, $S$ is the overlap matrix and $E$ is the energy matrix containing the orbital energies $\epsilon_i$.

The Fock matrix elements are,

\begin{gather}
{\begin{split}
F_{uv}=\langle\chi_u \mid f \mid \chi_v \rangle = \langle u \mid f \mid v \rangle \\
= H_{uv}+ \sum_{s,t} P_{st}[\langle us \mid vt \rangle - \frac{ \langle us \mid tv \rangle}{2} ]\hspace{0.2cm},
\end{split}}
\end{gather}

$P$ is the density matrix,

\begin{gather}
P_{uv}=\sum_{a}C_{ua}C_{va}n_a\hspace{0.2cm},
\end{gather}

\noindent where $n_a$ is the occupation number.

The overlap matrix elements are,

\begin{gather}
S_{uv}=\langle\chi_u\mid\chi_v\rangle=\langle u \mid  v \rangle\hspace{0.2cm}.
\end{gather}

A self-consistent solution of the previous equations is known as the Hartree-Fock, Self-Consistent Field (SCF) methodology.

These {\it ab initio} expressions for the Fock matrix will be used below to explain additional details of the methods used in this research.

