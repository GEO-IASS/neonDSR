\chapter{REFERENCES}
\section{Manually}
You can do your citations and references in LaTeX manually. However, if you do, you lose one of the top reasons for using LaTeX in the first place. Create a bibliography.tex file using the ufsampleETD.bbl file as a guide and include it using the \verb=% This setup has you inputting the bibliography manually, this can be done away with by using
% BiBTeX, but that has not been used/implemented in this template, so, you are free/encouraged
% to use it to make your life much easier, but we are still in the process of integrating BiBTeX
\begin{thebibliography}{100}
\pagestyle{plain}
\bibitem{book:guide}
The~Editorial Office of the Graduate School,
\newblock \emph{Guide for Preparing Theses and Dissertations},
\newblock The University of Florida, \emph{Revised} 1998.

\bibitem{abdulra.m.sheikh.auh.ea:94}
M.~Abdulrahan, A.~U.~H. Sheikh, and D.~D. Falconer,
\newblock ``Decision feedback equalization for {CDMA} in indoor wireless
  communications,''
\newblock \emph{IEEE Journal on Selected Areas in Communications}, vol. 12, no.
  4, pp. 698--706, May 1994.

\bibitem{barbosa.an.miller.sl:98}
A.~N. Barbosa and S.~L. Miller,
\newblock ``Adaptive detection of {DS/CDMA} signals in fading channels,''
\newblock \emph{IEEE Transactions on Communications}, vol. 46, no. 1, pp.
  115--124, Jan. 1998.

\bibitem{barghou.rt.stuber.gl:92}
R.~T. Barghouthi and G.~L. St{\"u}ber,
\newblock ``Sequence acquisition techniques for {CDMA} systems employing
  {Kasami} sequences,''
\newblock in \emph{Proceedings~IEEE Military Communications Conference}, San
  Diego, CA, 1992, vol.~2, pp. 633--637.

\bibitem{barghou.rt.stuber.gl:94}
R.~T. Barghouthi and G.~L. St{\"u}ber,
\newblock ``Rapid sequence acquisition for {DS/CDMA} systems employing {Kasami}
  sequences,''
\newblock \emph{IEEE Transactions on Communications}, vol. 42, no. 2/3/4, pp.
  1957--1968, Feb./Mar./Apr. 1994.

\bibitem{bensley.se.aazhang.b:96}
S.~E. Bensley and B.~Aazhang,
\newblock ``Subspace-based channel estimation for code division multiple access
  communication systems,''
\newblock \emph{IEEE Transactions on Communications}, vol. 44, no. 8, pp.
  1009--1020, Aug. 1996.

\bibitem{bensley.se.aazhang.b:98}
S.~E. Bensley and B.~Aazhang,
\newblock ``{Maximum}-{Likelihood} synchronization of a single user for
  code-division multiple-access communication systems,''
\newblock \emph{IEEE Transactions on Communications}, vol. 46, no. 3, pp.
  392--399, Mar. 1998.

\bibitem{bershad.nj.qu.lz:89}
N.~J. Bershad and L.~Z. Qu,
\newblock ``On the probability density function of the {LMS} adaptive filter
  weights,''
\newblock \emph{IEEE Transactions on Acoustics, Speech and Signal Processing},
  vol. 37, no. 1, pp. 43--56, Jan. 1989.

\bibitem{braun.wr:97}
W.~R. Braun,
\newblock ``{PN} acquisition and tracking performance in {DS/CDMA} systems with
  symbol-length spreading sequences,''
\newblock \emph{IEEE Transactions on Communications}, vol. 45, no. 12, pp.
  1595--1601, Dec. 1997.

\bibitem{buehrer.rm.kaul.a.ea:96}
R.~M. Buehrer, A.~Kaul, S.~Striglis, and B.~D. Woerner,
\newblock ``Analysis of {DS-CDMA} parallel interference cancellation with phase
  and timing errors,''
\newblock \emph{IEEE Journal on Selected Areas in Communications}, vol. 14, no.
  8, pp. 1522--1535, Oct. 1996.

\bibitem{buehrer.rm.woerner.bd:95}
R.~M. Buehrer and B.~D. Woerner,
\newblock ``Analysis of adaptive multistage interference cancellation for
  {CDMA} using an improved {Gaussian} approximation,''
\newblock in \emph{Proceedings~IEEE Military Communications Conference}, San
  Diego, CA, 1995, vol.~3, pp. 1195--1199.

\bibitem{chang.rw.sriniva.r:80}
R.~W. Chang and R.~Srinivasagopalan,
\newblock ``Carrier recovery for data communication systems with adaptive
  equalization,''
\newblock \emph{IEEE Transactions on Communications}, vol. COM--28, no. 8, pp.
  1142--1153, Aug. 1980.

\bibitem{clarke.rh:68}
R.~H. Clarke,
\newblock ``A statistical theory of mobile-radio reception,''
\newblock \emph{The Bell System Technical Journal}, vol. 47, no. 6, pp.
  957--1000, July-Aug. 1968.

\bibitem{couch-ii.lw:97}
L.~W. Couch~II,
\newblock \emph{Digital and Analog Communication Systems},
\newblock Prentice-Hall, 5th edition, 1997.

\bibitem{davies.rb:73}
R.~B. Davies,
\newblock ``Numerical inversion of a characteristic function,''
\newblock \emph{Biometrika}, vol. 60, no. 2, pp. 415--417, 1973.

\bibitem{davies.rb:80}
R.~B. Davies,
\newblock ``The distribution of a linear combination of $\chi^2$ random
  variables,''
\newblock \emph{Applied Statistics}, vol. 29, no. 3, pp. 323--333, 1980.

\bibitem{douglas.sc.meng.thy:92}
S.~C. Douglas and T.~H.-Y. Meng,
\newblock ``Exact expectation analysis of the {LMS} adaptive filter without the
  independence assumption,''
\newblock in \emph{Proceedings~IEEE International Conference on Acoustics,
  Speech and Signal Processing}, San Francisco, CA, 1992, vol.~4, pp. 61--64.

\bibitem{douglas.sc:93}
S.~C. Douglas,
\newblock ``Exact expectation analysis of the {LMS} adaptive filter for
  correlated {Gaussian} input data,''
\newblock in \emph{Proceedings~IEEE International Conference on Acoustics,
  Speech and Signal Processing}, Minneapolis, MN, 1993, vol.~3, pp. 519--522.

\end{thebibliography}
= command. NOTE: This method will not allow you to adjust the format of your bibliography by changing the .bst file called in the main document.
\section{BibTeX}
We use the package natbib in the template. \citep{Dill93} It is used by a large number of journals and offers the widest variety of citation and reference listing options with the least amount of overhead and/or complexity. Determine whether you prefer numbered or un-numbered reference listing. Go to the packages.tex file and make sure your preference is uncommented. Comment (or delete) the other option and you're ready to select a bibliography style.\citep{Adams89} This is done with the \verb=\bibliographystyle{bibstyle}= command. We have included several basic reference styles(the .bst file types). The differences are noted in the following list. Note that the plain style does not change in the either the numbered or author-yearnumbers. \citep{andr:2000}
\begin{itemize}
\item Numbered References (using the numbered, sort and compress natbib option)
\begin{itemize}
\item plain: numbered citations-brackets, numbered reference list in alphabetical order, full first name last name.
\item ufinit: numbered citations-brackets, numbered reference list in citation order, initials and last name.
\item plainnat: numbered citations-brackets, numbered reference list in alphabetical order, full first name last name.
\item abbrvnat: numbered citations-brackets, numbered reference list in alphabetical order, initials and last name.
\item unsrtnat: numbered citations-brackets, numbered reference list in citation order, full first name last name.
\item chicagoReedWeb: numbered citations-brackets, numbered reference list in alphabetical order, last name, full first name.
\item apa-good: numbered citations-brackets, numbered reference list in alphabetical order, last name, first and middle initials.
\end{itemize}
\item Un-numbered References (using the authoryear natbib option)
\begin{itemize}
\item plain: numbered citations-brackets, numbered reference list in alphabetical order, full first name last name.
\item ufinit: numbered citations-parentheses, numbered reference list in citation order, initials and last name.
\item plainnat: author-year citations-brackets, un-numbered reference list in alphabetical order, full first name, last name.
\item abbrvnat: author-year citations-brackets, un-numbered reference list in alphabetical order, initial first name, last name.
\item unsrtnat: author-year citations-brackets, un-numbered reference list in citation order, initial first name, last name.
\item chicagoReedWeb: author-year citations-parentheses, un-numbered reference list in alphabetical order, last name, first name, line replaces repeated author.
\item apa-good: author-year citations-parentheses, un-numbered reference list in alphabetical order, last name, first initials.
\end{itemize}
\end{itemize}
We have included some additional .bst files that may or may not be useful. Please note! We use several citation commands to illustrate the different results and the bibliography in this document cannot be used as an example of any specific reference system. It is YOUR responsibility to  determine the reference style of the journal you want to emulate and obtain the necessary .bst files to emulate that style.
\section{Footnotes}
To create footnotes you simply use the \verb=\footnote{text}= command.\footnote{See, it really does work} This works well except when it is used inside any environment that produces (or can produce) a box.\footnote{I've seen examples of documents that failed to compile simply because of the placement of a footnote command.} In that case the manual suggests you use \verb=\footnotemark= inside the environment and just outside the environment use \verb=\foottext{text}= to define the text to accompany the footnote mark. \citep{Boek97} Frankly, I have had some trouble making this option work, particularly in the longtable package and recommend that you avoid footnotes inside any ``environment'' other than the normal text mode if at all possible.\footnote{Footnotes should be single-spaced with a space between each note. They should also start re-numbering at one each chapter!} I'm repeating our example of the decimal alignment work-around to illustrate the footnotemark - footnotetext example.


\begin{table}[h!]
\caption[This is an example of an optional caption - this will appear in the list of tables - the caption in curly brackets will appear with the table]{You shouldn't have two tables (or figures) with the same caption. Even if there's a small difference well into the caption it is best to put the difference first to uniquely identify the table or figure.}\label{tb3}
  \begin{tabular}{c r@{.}l}
 \hline
 Category & \multicolumn{2}{c}{Result}\\
 \hline
  first & 3&14159\footnotemark\\
  second & 16&2\\
  third & 123&456\\
  \hline
  \end{tabular}
\end{table}
\footnotetext{This is an example of the footnotemark process for box creating environments - I sure hope this works!} 