\chapter{GENERAL THESIS TIPS}
\begin{itemize}
\item	BACK UP YOUR THESIS. Often you will not realize for days or weeks that important paragraph or page is missing. Make recovery as easy as possible by keeping a dated backup of each writing session. Then copy those backups to at least two locations other than your hard drive: your home server, gmail account, thumb drive, the options are wide and numerous. There is no excuse for not backing up the most important document of your academic career.
\item	Start your bibliographic database the day you start reading. Keep it up to date and annotate it, so you know where it came from (library, Internet, public library, professor), whether you've read it, and where you want to cite it. This will make the writing process less frustrating and creating the bibliography seamless.
\item	Think of thesis formatting as a form of productive procrastination. Please don't put it off until the last week.
\item	BACK UP. No, seriously. It's not ``if'' your hard drive fails, it's ``when.'' Not to scare you or anything, but it's a good habit, like buckling your seat belt or not leaving your laptop unattended. You really don't want to wish you had taken that small precaution.
\item	Keep the editable original of each graphic you want to include in your thesis in one folder. Later you may need to change a graphic quickly and having the editable original makes it easy. For graphs, keep the original Excel/JMP/Stata document, not a PDF. For photographs, keep a high resolution copy. For drawings and illustrations, keep the original document.
\item	Use the timesaving benefits of LaTeX from the first day. Cross references can refer to tables, graphics, and chapters so you do not have to update references as your thesis changes. Use comments to make notes about what needs to added or changed.
\item	Enjoy the experience! And get some sleep, food and relaxation on occasion. Thousands of people did this before you; you can do this too.
\end{itemize}
