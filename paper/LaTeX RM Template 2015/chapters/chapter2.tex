\chapter{KNOWN ISSUES}
\section{Common Problems}
\subsection{Can't Find dvipdfm}

If you're using a MacIntosh and typesetting your file results in an error stating it can't find the file dvipdfm. That option is listed in three places and will need to be changed. The preamble of the main file, and two places in the packages.tex file (the graphix package and the hyperref package). Replace dvipdfm with dvipdfmx and typeset using XeLaTeX and that should take care of the problem.

\subsection{Prime Notation}

If you've placed your own content into the template and it fails to compile the most common reason is the use of prime notation in math mode. If you say anything like A' while in the math mode it generates a conflict that will cause the rest of the document to stop compiling. The solution is to use an alternative method of producing the same notation. Replace \verb=A'=  with \verb=A^{\prime}= and you will avoid this issue. Since prime notation is quite common we defined a shortcut for this command which allows you to replace \verb=A'= with \verb=A\p.= \citep{Brady01}

\subsection{Long (and/or Wide) Tables}

Another problem in LaTeX is the inability to handle long tables. While there are some packages that address this problem none of them quite fit the Editorial Office guidelines. The caption is not repeated but we do need "Table x-y. Continued" on each subsequent page and a repeat of the column headings on each page as well. The following table is the best example of the correct format I can produce. The disadvantage of this method is that much of it is manually set up and changes in the text will cause changes in the table. \citep{Sato91}

For example, at one time the following table was perfectly positioned at the beginning of a page - right after a full page of text. Now the footnote appears on the page with text that is two full pages before the footnote mark! For best results avoid the use of footnotemark and footnotetext commands inside of tables and try to keep your footnotes outside of floats whenever possible.



\begin{table}
\caption{Feasible triples for highly variable Grid, MLMMH.} \label{tbl1}
\begin{tabular}{r l l}
\hline {{Time (s)}} & {{Triple chosen}} & {{Other feasible triples}} \\ \hline
0 & (1, 11, 13725) & (1, 12, 10980), (1, 13, 8235), (2, 2, 0), (3, 1, 0) \\
2745 & (1, 12, 10980) & (1, 13, 8235), (2, 2, 0), (2, 3, 0), (3, 1, 0) \\
5490 & (1, 12, 13725) & (2, 2, 2745), (2, 3, 0), (3, 1, 0) \\
8235 & (1, 12, 16470) & (1, 13, 13725), (2, 2, 2745), (2, 3, 0), (3, 1, 0) \\
10980 & (1, 12, 16470) & (1, 13, 13725), (2, 2, 2745), (2, 3, 0), (3, 1, 0) \\
13725 & (1, 12, 16470) & (1, 13, 13725), (2, 2, 2745), (2, 3, 0), (3, 1, 0) \\
16470 & (1, 13, 16470) & (2, 2, 2745), (2, 3, 0), (3, 1, 0) \\
19215 & (1, 12, 16470) & (1, 13, 13725), (2, 2, 2745), (2, 3, 0), (3, 1, 0) \\
21960 & (1, 12, 16470) & (1, 13, 13725), (2, 2, 2745), (2, 3, 0), (3, 1, 0) \\
24705 & (1, 12, 16470) & (1, 13, 13725), (2, 2, 2745), (2, 3, 0), (3, 1, 0) \\
27450 & (1, 12, 16470) & (1, 13, 13725), (2, 2, 2745), (2, 3, 0), (3, 1, 0) \\
30195 & (2, 2, 2745) & (2, 3, 0), (3, 1, 0) \\
32940 & (1, 13, 16470) & (2, 2, 2745), (2, 3, 0), (3, 1, 0) \\
35685 & (1, 13, 13725) & (2, 2, 2745), (2, 3, 0), (3, 1, 0) \\
38430 & (1, 13, 10980) & (2, 2, 2745), (2, 3, 0), (3, 1, 0) \\
41175 & (1, 12, 13725) & (1, 13, 10980), (2, 2, 2745), (2, 3, 0), (3, 1, 0) \\
43920 & (1, 13, 10980) & (2, 2, 2745), (2, 3, 0), (3, 1, 0) \\
46665 & (2, 2, 2745) & (2, 3, 0), (3, 1, 0) \\
49410 & (2, 2, 2745) & (2, 3, 0), (3, 1, 0) \\
52155 & (1, 12, 16470) & (1, 13, 13725), (2, 2, 2745), (2, 3, 0), (3, 1, 0) \\
54900 & (1, 13, 13725) & (2, 2, 2745), (2, 3, 0), (3, 1, 0) \\
57645 & (1, 13, 13725) & (2, 2, 2745), (2, 3, 0), (3, 1, 0) \\
60390 & (1, 12, 13725) & (2, 2, 2745), (2, 3, 0), (3, 1, 0) \\
63135 & (1, 13, 16470) & (2, 2, 2745), (2, 3, 0), (3, 1, 0) \\
65880 & (1, 13, 16470) & (2, 2, 2745), (2, 3, 0), (3, 1, 0) \\
68625 & (2, 2, 2745) & (2, 3, 0), (3, 1, 0) \\
71370 & (1, 13, 13725) & (2, 2, 2745), (2, 3, 0), (3, 1, 0) \\
74115 & (1, 12, 13725) & (2, 2, 2745), (2, 3, 0), (3, 1, 0) \\
76860 & (1, 13, 13725) & (2, 2, 2745), (2, 3, 0), (3, 1, 0) \\
79605 & (1, 13, 13725) & (2, 2, 2745), (2, 3, 0), (3, 1, 0) \\
82350 & (1, 12, 13725) & (2, 2, 2745), (2, 3, 0), (3, 1, 0) \\
85095 & (1, 12, 13725) & (1, 13, 10980), (2, 2, 2745), (2, 3, 0), (3, 1, 0) \\
87840 & (1, 13, 16470) & (2, 2, 2745), (2, 3, 0), (3, 1, 0) \\
90585 & (1, 13, 16470) & (2, 2, 2745), (2, 3, 0), (3, 1, 0) \\
93330 & (1, 13, 13725) & (2, 2, 2745), (2, 3, 0), (3, 1, 0) \\
96075 & (1, 13, 16470) & (2, 2, 2745), (2, 3, 0), (3, 1, 0) \\
98820 & (1, 13, 16470) & (2, 2, 2745), (2, 3, 0), (3, 1, 0) \\
101565 & (1, 13, 13725) & (2, 2, 2745), (2, 3, 0), (3, 1, 0) \\
104310 & (1, 13, 16470) & (2, 2, 2745), (2, 3, 0), (3, 1, 0) \\
107055 & (1, 13, 13725) & (2, 2, 2745), (2, 3, 0), (3, 1, 0) \\
109800 & (1, 13, 13725) & (2, 2, 2745), (2, 3, 0), (3, 1, 0) \\
112545 & (1, 12, 16470) & (1, 13, 13725), (2, 2, 2745), (2, 3, 0), (3, 1, 0) \\
\hline
\end{tabular}
\end{table}

\begin{table}[h!t!]
\begin{tabular}{r l l}
\multicolumn{3}{l}{Table \ref{tbl1}. Continued}\\%
\hline {{Time (s)}} & {{Triple chosen}} & {{Other feasible triples}} \\ \hline
115290 & (1, 13, 16470) & (2, 2, 2745), (2, 3, 0), (3, 1, 0) \\
118035 & (1, 13, 13725) & (2, 2, 2745), (2, 3, 0), (3, 1, 0) \\
120780 & (1, 13, 16470) & (2, 2, 2745), (2, 3, 0), (3, 1, 0) \\
123525 & (1, 13, 13725) & (2, 2, 2745), (2, 3, 0), (3, 1, 0) \\
126270 & (1, 12, 16470) & (1, 13, 13725), (2, 2, 2745), (2, 3, 0), (3, 1, 0) \\
129015 & (2, 2, 2745) & (2, 3, 0), (3, 1, 0) \\
131760 & (2, 2, 2745) & (2, 3, 0), (3, 1, 0) \\
134505 & (1, 13, 16470) & (2, 2, 2745), (2, 3, 0), (3, 1, 0) \\
137250 & (1, 13, 13725) & (2, 2, 2745), (2, 3, 0), (3, 1, 0) \\
139995 & (2, 2, 2745) & (2, 3, 0), (3, 1, 0) \\
142740 & (2, 2, 2745) & (2, 3, 0), (3, 1, 0) \\
145485 & (1, 12, 16470) & (1, 13, 13725), (2, 2, 2745), (2, 3, 0), (3, 1, 0)\\%
148230 & (2, 2, 2745) & (2, 3, 0), (3, 1, 0) \\
150975 & (1, 13, 16470) & (2, 2, 2745), (2, 3, 0), (3, 1, 0) \\
153720 & (1, 12, 13725) & (2, 2, 2745), (2, 3, 0), (3, 1, 0) \\
156465 & (1, 13, 13725) & (2, 2, 2745), (2, 3, 0), (3, 1, 0) \\
159210 & (1, 13, 13725) & (2, 2, 2745), (2, 3, 0), (3, 1, 0) \\
161955 & (1, 13, 16470) & (2, 2, 2745), (2, 3, 0), (3, 1, 0) \\
164700 & (1, 13, 13725) & (2, 2, 2745), (2, 3, 0), (3, 1, 0) \\
\hline
\end{tabular}
\end{table}

%% The blank lines at the end of the table are there to force the table to the top of the page.
%% Not a necessity but with repeated tables I feel they should start at the top of the page.

\subsection{Page Size}

When installing MiKTeX one of the set up questions asks for the default paper size. Unless you select "Letter" it will default to A4. Although the Template specifies letter size in several locations an installation with an A4 default will create a document using A4 paper size. The Editors will notice that your margins are incorrect and comment on this during first submission (it probably won't cause rejection on first submission). However, it will cause a problem for final approval. The only solution I know of is to re-install MiKTex and give the "Letter" size as the default paper size. The alternative would be to compile your document on an AT Lab computer which should have all been set to letter size when the Hard Drive Image was created.

\subsection{Single Appendix}

Since LaTeX numbers everything automatically there is an interesting problem that occurs whenever an author wants to create a document with a single appendix. When this happens, the Editorial Office states that instead of ``numbering'' the appendix as ``A'' the word ``APPENDIX'' should appear in the Table of Contents on the same line as the appendix title without the letter ``A.'' This can be done by suppressing the chapter numbers but then anything that needs to be numbered (Tables, Figures, and/or equations is numbered as a continuation of the previous chapter. This is fine if there's nothing to number, but most of the time this is not the case.

The file appendix.tex is used to control the number of appendices through the counter ``noa.'' Set the value to 1 if there is a single appendix any larger value will work for multiple appendices. If the value is 1 input only appendixA, if the value is two or more allow the other appendices to be input (adding input statements if needed). The beginning of appendixA has another ifthenelse statement regarding the number of appendices and then demonstrates how a landscaped page is inserted as the first page of an appendix. \citep{vanBruggen98}

\subsection{Decimal Alignment}

One of the strangest facts about LaTeX is that it doesn't have a simple method of aligning numbers in a table on the decimal point. The workaround is to create two separate columns align the first to the right and the second to the left and set the separator to a decimal point. This will give the illusion of a decimally aligned column. \citep{Chwang74}

\begin{table}[h!]
\caption{How to align decimals in a numerical column}\label{tb1}
  \begin{tabular}{p{3 in} r@{.}l c}
 \hline
 Category & \multicolumn{2}{c}{Result}  & \hspace{2.2 in} \\
 \hline
  first & 3&14159 & \hspace{2.2 in} \\
  second & 16&2 & \hspace{2.2 in} \\
  third & 123&456 & \hspace{2.2 in} \\
  \hline
  \end{tabular}
\end{table}

%\begin{table}[h!]
%\caption{How to align decimals in a numerical column}\label{tb1}
%  \begin{tabularx}{6.5 in}{X X r@{.}l X }
% \hline
% Category & & \multicolumn{2}{c}{Result} & \hfill\\
% \hline
%  first &  & 3&14159 & \hfill \\
%  second &  & 16&2 & \hfill \\
%  third &  & 123&456 & \hfill \\
%  \hline
%  \end{tabularx}
%\end{table}

\noindent As you can see in Table \ref{tb1} the result is an illusion of a decimally aligned column exactly as preferred by the Editorial Office. If done carefully nobody will ever know your dirty secret!


\section{Images That Do Not Show}

If you're trying to use the package psfrag you must change the method of compilation for it to render your images correctly. In the main file, change dvipdfm to dvips. Make this same change in the packages file with both the graphix and hyperref package options. You then must compile in the folowing manner:

\begin{enumerate}

\item latex filename
\item latex filename
\item bibtex filename (if needed)
\item latex filename
\item latex filename  (these latex commands are only needed if the bibtex command was used)
\item dvips filename
\item ps2pdf filename
\end{enumerate}
This can still have a negative effect on the hyperref package and result in broken links and/or incorrect margins in the TOC, LOT and LOF. 